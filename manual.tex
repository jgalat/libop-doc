\documentclass[12pt,a4paper,final]{article}
\usepackage[utf8]{inputenc}
\usepackage[spanish]{babel}
\usepackage{amsmath}
\usepackage{amsfonts}
\usepackage{amssymb}
\usepackage{imakeidx}
\usepackage[hidelinks]{hyperref}
\usepackage[none]{hyphenat}
\usepackage{alltt}
\author{}
\title{LibOP \\ Documentación técnica}
\makeindex
\date{}

\begin{document}
\maketitle

\tableofcontents

\clearpage

\section{Introducción}


\subsection{Descripción}
	%TODO que hace, en que está hecha? etc..
	
	
\section{Compilación}
	En subsecciones siguientes se listan los requerimientos a cumplir y las instrucciones
	a correr necesarias para la correcta compilación de la librería en sistemas basados
	en Linux y en Windows. 	 
	
	\subsection{Linux}
		
		\subsubsection{Requerimientos}
			En sistemas derivados de Debian el único requerimiento es el conjunto de 
			herramientas que vienen incluídas en el paquete \texttt{build-essential},
			el cual puede ser instalado desde los repositorios
			oficiales utilizando el comando:
			
			\begin{alltt}
				sudo apt-get install build-essential
			\end{alltt}	
			
		\subsubsection{Instrucciones}
			Ubicado en la carpeta del proyecto clonado podremos compilar la librería corriendo:
			
			\begin{alltt}
				cd build && make
			\end{alltt}

			Finalizado el comando obtendremos la librería compartida \texttt{libop.so}.

	\subsection{Windows}
		\subsubsection{Requerimientos}
			En sistemas que utilicen Windows 7, 8.1 o 10 los siguientes requerimientos
			son necesarios (a menos que se indique lo contrario) para la correcta compilación:
			\begin{itemize}
				\item \href{https://git-scm.com/download/win}{Git for Windows} (Opcional)
				\item \href{http://gnuwin32.sourceforge.net/packages/make.htm}{GNU Make}
				\item \href{https://sourceforge.net/projects/mingw-w64/}{Mingw-w64}
			\end{itemize}
		\subsubsection{Instrucciones}
			Seguir las siguientes instrucciones.
			
			\begin{enumerate}
				\item Instalar Git for Windows (opcional).
				\item Instalar GNU Make.
					\begin{itemize}
						\item Añadir la carpeta \texttt{GnuWin32/bin} a la variable de entorno \texttt{PATH}.
					\end{itemize}
				\item Instalar Mingw-w64 con las opciones (7.1.0, x86\_64, posix, seh, 0). El número de versión puede variar.
					\begin{itemize}
						\item Añadir la carpeta \texttt{mingw-w64/x86\_64-7.1.0-posix-seh-rt\_05-rev0/mingw64/bin} a la variable de entorno \texttt{PATH}.
					\end{itemize}
				\item Clonar el proyecto mediante Git.
				\item Ubicado en la carpeta del proyecto correr: 
						\begin{alltt}
							cd build \&\& make OS=win32
						\end{alltt}			
			\end{enumerate}
		
			Finalizado todo obtendremos la librería compartida \texttt{libop.dll}.
			
	\subsection{Opciones de compilación}
		Se listan a continuación opciones que se pueden pasar al comando \texttt{make}.
		
		\begin{description}
			\item [\texttt{make OS=\textbf{os}}] Especifica el sistema para el cual se desea compilar la librería.
				Sus opciones son \texttt{unix} o \texttt{win32}. Siendo \texttt{unix} la opción por defecto.
			\item [\texttt{make VERBOSE=bool}] Caso sea \texttt{True} la librería compilada incluirá código 
				necesario para la impresión de mensajes de error a través de la salida estandar de errores. Ésto
				puede ser útil para la búsqueda de errores. Su opción por defecto es \texttt{False}.
			\item [\texttt{make export\_all}] Creará una carpeta \texttt{export} en la raíz del proyecto la 
				cual incluirá la librería compilada y las cabeceras C a ser incluídas para su uso. Caso la
				plataforma objetivo sea Windows, deberemos especificar también la opcion \texttt{OS=win32}.
		\end{description}

\section{Uso}
	\section{C}

\section{Implementación}

	\subsection{Guía de estructuras}

	\subsection{Guía de funciones}
		\subsubsection{cabecera.h, etc, funcs abajo}

\section{Extensión de la librería?}
	\subsection{Como agregar nuevos pms?, etc}

\section{Ejemplo}

\section{Biblio???}

\printindex

\end{document}