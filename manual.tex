\documentclass[12pt,a4paper,final]{article}
\usepackage[utf8]{inputenc}
\usepackage[spanish]{babel}
\usepackage{amsmath}
\usepackage{amsfonts}
\usepackage{amssymb}
\usepackage{imakeidx}
\usepackage[hidelinks]{hyperref}
\usepackage[none]{hyphenat}
\usepackage{alltt}
\usepackage{bold-extra}
\author{}
\title{LibOP \\ Documentación técnica}
\makeindex
\date{}

\begin{document}
\maketitle

\tableofcontents

\clearpage

\section{Introducción}


\subsection{Descripción}
	%TODO que hace, en que está hecha? etc..
	
	
\section{Compilación}
	En subsecciones siguientes se listan los requerimientos a cumplir y las instrucciones
	a correr necesarias para la correcta compilación de la librería en sistemas basados
	en Linux y en Windows. 	 
	
	\subsection{Linux}
		
		\subsubsection{Requerimientos}
			En sistemas derivados de Debian el único requerimiento es el conjunto de 
			herramientas que vienen incluídas en el paquete \texttt{build-essential},
			el cual puede ser instalado desde los repositorios
			oficiales utilizando el comando:
			
			\begin{alltt}
				sudo apt-get install build-essential
			\end{alltt}	
			
		\subsubsection{Instrucciones}
			Ubicado en la carpeta del proyecto clonado podremos compilar la librería corriendo:
			
			\begin{alltt}
				cd build && make
			\end{alltt}

			Finalizado el comando obtendremos la librería compartida \texttt{libop.so}.

	\subsection{Windows}
		\subsubsection{Requerimientos}
			En sistemas que utilicen Windows 7, 8.1 o 10 los siguientes requerimientos
			son necesarios (a menos que se indique lo contrario) para la correcta compilación:
			\begin{itemize}
				\item \href{https://git-scm.com/download/win}{Git for Windows} (Opcional)
				\item \href{http://gnuwin32.sourceforge.net/packages/make.htm}{GNU Make}
				\item \href{https://sourceforge.net/projects/mingw-w64/}{Mingw-w64}
			\end{itemize}
		\subsubsection{Instrucciones}
			Seguir las siguientes instrucciones.
			
			\begin{enumerate}
				\item Instalar Git for Windows (opcional).
				\item Instalar GNU Make.
					\begin{itemize}
						\item Añadir la carpeta \texttt{GnuWin32/bin} a la variable de entorno \texttt{PATH}.
					\end{itemize}
				\item Instalar Mingw-w64 con las opciones (7.1.0, x86\_64, posix, seh, 0). El número de versión puede variar.
					\begin{itemize}
						\item Añadir la carpeta \texttt{mingw-w64/x86\_64-7.1.0-posix-seh-rt\_05-rev0/mingw64/bin} a la variable de entorno \texttt{PATH}.
					\end{itemize}
				\item Clonar el proyecto mediante Git.
				\item Ubicado en la carpeta del proyecto correr: 
						\begin{alltt}
							cd build \&\& make OS=win32
						\end{alltt}			
			\end{enumerate}
		
			Finalizado todo obtendremos la librería compartida \texttt{libop.dll}.
			
	\subsection{Opciones de compilación}
		Se listan a continuación opciones que se pueden pasar al comando \texttt{make}.
		
		\begin{description}
			\item [\texttt{make OS=\textbf{os}}] Especifica el sistema para el cual se desea compilar la librería.
				Sus opciones son \texttt{unix} o \texttt{win32}. Siendo \texttt{unix} la opción por defecto.
			\item [\texttt{make VERBOSE=bool}] Caso sea \texttt{True} la librería compilada incluirá código 
				necesario para la impresión de mensajes de error a través de la salida estandar de errores. Ésto
				puede ser útil para la búsqueda de errores. Su opción por defecto es \texttt{False}.
			\item [\texttt{make export\_all}] Creará una carpeta \texttt{export} en la raíz del proyecto la 
				cual incluirá la librería compilada y las cabeceras C a ser incluídas para su uso. Caso la
				plataforma objetivo sea Windows, deberemos especificar también la opcion \texttt{OS=win32}.
		\end{description}

\section{Uso}
	\subsection{C/C++}
		Las signaturas de las funciones exportadas por la librería se encuentran en las cabeceras
		exportadas por la opción \texttt{export\_all}. Para hacer uso de la librería sólo deberemos incluír
		en el código la cabecera \texttt{libop.h} ubicada en \texttt{include} y compilar de manera que se linkee
		a la librería compilada. Ejemplo:
		
		\begin{alltt}
			\textbf{CC} -I\textbf{include} -L\textbf{lib} ejemplo.c -lop
		\end{alltt}
		
		Dónde \texttt{CC} es el compilador a utilizar (gcc, g++, etc.), \texttt{include} es la carpeta dónde
		se encuentran las cabeceras y \texttt{lib} es la carpeta dónde se encuentra la librería compilada.
		Notar el uso de la bandera \texttt{-lop}, la cual es necesaria.
		
	\subsection{C\#}
		En el siguiente enlace: \url{https://www.github.com/jgalat/libop_CS}, se encuentra alojada
		una solución hecha en Visual Studio la cual actúa como interfaz para invocar a las funciones
		de la librería desde C\#. La solución requiere de la librería compilada utilizando la bandera
		\texttt{OS=win32}, es decir \texttt{libop.dll}.
		
\section{Implementación}

	\subsection{Guía de estructuras}
		A continuacióin se describen las estructuras incluídas en la librería.
		
		\begin{description}
			\item [\texttt{time\_period}] Representa un periodo de tiempo. A partir de la definicion de 
				la cantidad de días activos que habrá en el año, se puede determinar el número
				de días activos que el periodo abarca.  
			\item [\texttt{dividend}] Representa un dividendo, el cual puede ser continuo o discreto.
			\item [\texttt{volatility}] Representa una volatilidad.
			\item [\texttt{risk\_free\_rate}] Representa una tasa libre de riesgo.
			\item [\texttt{pm\_settings}] Estructura que almacena configuraciones que se pueden
				especificar para ser usadas por un \texttt{pricing\_method}. Más adelante se describiran
				sus campos, valores por defecto y como cambiarlos.
			\item [\texttt{pricing\_method}] Estructura que contiene métodos para realizar el cálculo
				de precios, griegas y volatilidades implícitas de opciones. Cada \texttt{pricing\_method}
				se encuentra programado de forma exclusiva para un tipo de ejercicio en particular,
				es decir, por ejemplo, no se podrá calcular el precio de una opción europea con un
			\texttt{pricing\_method} pensado para americanas. Para el cálculo hara uso de los datos
				de la opción, dividendo, volatilidad, tasa de riesgo y (caso se haya especificado) un conjunto
				de configuraciones diferentes a las por defecto.
			\item [\texttt{result}]	Estrctura en la cual se almacenaran los resultados calculados por
				un \texttt{pricing\_method}. Una vez realizado el cálculo por éste, se podrá consultar para
				su obtención.
			\item [\texttt{option}] Representa una opción, es decir, especifica su tipo (Call o Put), su
				tipo de ejercicio (Europeo o Americano), su periodo hasta la expiración y su precio de
				strike. Hace uso de \texttt{pricing\_method} para el cálculo de su precio, griegas y
				volatilidades implicitas.
		\end{description}

	\subsection{Guía de funciones y tipos}
		A continuación se detallan las funciones y tipos definidos específicos de cada estructura
		listados según la cabecera en la que se encuentran.
		
		\subsubsection{dividend.h}
			\begin{description}
			
				\item [\texttt{dividend\_type}]
					Es el tipo enumerado de los dividendos. Sus elementos son
					\texttt{DIV\_DISCRETE} y \texttt{DIV\_CONTINUOUS}.
					
				\item [\texttt{dividend}]
					Tipo de la estructura dividendo.
				
				\item [\texttt{dividend new\_continuous\_dividend(double continuous\_dividend)}]
					Crea una estructura del tipo dividendo. Su tipo será continuo.
					Toma como parametro el dividendo continuo en decimal (ej.: 10\% = 0.1).
					En caso de error retorna \texttt{NULL}
					
				\item [\texttt{dividend new\_discrete\_dividend()}]
					Crea una estructura del tipo dividendo. Su tipo será discreto.
					La estructura creada no posee dividendos, es decir, caso sea utilizado
					actuará como un dividendo continuo 0. Más adelante se define como poblar
					la estructura con fechas y pagos.
					En caso de error retorna \texttt{NULL}.
					
				\item [\texttt{void delete\_dividend(dividend d)}]
					Destruye y libera la memoria utilizada por un dividendo.
					
				\item [\texttt{dividend\_type div\_get\_type(dividend d)}]
					Devuelve el tipo del dividendo dado como parámetro.
					
				\item [\texttt{double div\_cont\_get\_val(dividend d)}]
					Devuelve el valor del dividendo continuo con el que se creo la estructura.
					En caso de error (ej.: el dividendo es discreto o \texttt{NULL}) retorna -1.
					
				\item [\texttt{int div\_disc\_get\_n(dividend d)}]
					Devuelve el número de pagos que hay almacenados en la estructura.
					En caso de error (ej.: el dividendo es continuo o \texttt{NULL} retorna -1.
					
				\item [\texttt{date *div\_disc\_get\_dates(dividend d)}]
					Devuelve el arreglo de fechas con pagos almacenados en la estructura.
					Las fechas se encuentran en tipo \texttt{date} el cual será explicado
					en la sección de \texttt{time\_period.h}. Pueden ser transformados a días
					utilizando el \texttt{time\_period} original.
					En caso de error (ej.: el dividendo es continuo o \texttt{NULL}) o si
					la lista es vacía, retorna \texttt{NULL}.
					
				\item [\texttt{double *div\_disc\_get\_ammounts(dividend d)}]
					Devuelve el arreglo de pagos almacenados en la estructura.
					La posición de un pago se corresponde con la de la fecha en esa posición.
					En caso de error (ej.: el dividendo es continuo o \texttt{NULL}) o si
					la lista es vacía, retorna \texttt{NULL}.
					
				\item [int div\_disc\_set\_dates(dividend d, time\_period tp, int size, ...)]
					Especifica el los días en los cuales habrá pagos. Los periodos son en días.
					Esta es la versión de argumentos variables, el número de pagos debe ser especificado
					en \texttt{size}. Se debe pasar el \texttt{time\_period} a utilizar. Los datos 
					especificados reemplazan los existentes.
					
					Ejemplo: Para establecer 2 fechas, una dentro de 100 días y otra dentro de 200
						se utilizaría
						\begin{alltt}
							div_disc_set_dates(d, tp, 2, 100, 200);
						\end{alltt}
					En caso de error retorna -1.
					
				\item [\texttt{int div\_disc\_set\_ammounts(dividend d, int size, ...)}]
					Especifica los montos a pagar. La posición en que se ubique el monto, corresponderá
					a la de la fecha. Esta es la versión de argumentos variables, el número de pagos debe ser
					especificado en \texttt{size}. Los datos especificados reemplazan los existentes.
					
					Ejemplo: Para establecer un único pago de \$5 se utilizaría
						\begin{alltt}
							div_disc_set_ammounts(d, 1, 5);
						\end{alltt}
					En caso de error retorna -1.
					
				\item [\texttt{int div\_disc\_set\_dates\_(dividend d, time\_period tp, int size, int *days)}]
					Versión de argumentos no variables de \texttt{div\_disc\_set\_dates(...)}.
					Se especifica un arreglo de días.
					Se comporta de la misma manera.
									
					Ejemplo: El siguiente código es equivalente al ejemplo dado en la función anterior.
						\begin{alltt}
							int days[2] = \{ 100, 200 \};
							div\_disc\_set\_dates\_(d, tp, 2, days);
						\end{alltt}				
				
				
				\item [\texttt{int div\_disc\_set\_ammounts\_(dividend d, int size, double *ammounts)}]
					Versión de argumentos no variables de \texttt{div\_disc\_set\_ammounts(...)}.
					Se especifica un arreglo con montos.
					Se comporta de la misma manera.	
					
					Ejemplo: El siguiente código es equivalente al ejemplo dado en la función anterior.
						\begin{alltt}
							double ammounts[1] = \{ 5 \};
							div\_disc\_set\_ammounts\_(d, 1, ammounts);
						\end{alltt}										
			\end{description}
			
\section{Extensión de la librería?}
	\subsection{Como agregar nuevos pms?, etc}

\section{Ejemplo}

\section{Biblio???}

\printindex

\end{document}